\pagenumbering{gobble}
\chapter*{Definitions}
\theoremstyle{definition}

\section{Functions of Several Variables}
\begin{mydef}
\normalfont A function \(f\) whose domain is \(\real^n\) or a subset of \(\real^n\), for \(n \ge 2\) and \(n \in \nat\), is called a function of several real variables.
\end{mydef}


\begin{mydef}
\normalfont For a function \(z = f(x, y)\): A vertical section is the graph of \(z = f(x, c)\) or \(z = f(c, y)\), for some constant \(c\). A level curve is the curve \(f(x, y) = c\), for some constant \(c\).
\end{mydef}


\section{Partial Differentiation}
\begin{mydef}
\normalfont The function \(f : \real^3 \to \real\) is said to have a partial derivative with respect to \(x\) at the point \((x_0, y_0, z_0\) if the following limit exists
%
\[\lim_{\Delta x \to 0}{\frac{f(x_0 + \Delta x, y_0, z_0) - f(x_0, y_0, z_0)}{\Delta x}}\]
%
which is called the \textbf{partial derivative} of \(f\) with respect to \(x\) at the point \((x_0, y_0, z_0)\), denoted as
%
\[\frac{\partial f(x_0, y_0, z_0)}{\partial x} \equiv \lim_{\Delta x \to 0}{\frac{f(x_0 + \Delta x, y_0, z_0) - f(x_0, y_0, z_0)}{\Delta x}}\]
\end{mydef}

\section{Differentiable Function}
\begin{mydef}
\normalfont The function \(f(x, y, z)\) is called differentiable at \((x_0, y_0, z_0)\) if \(\Delta f = f(x, y, z) - f(x_0, y_0, z_0)\) can be expressed as
%
\[\Delta f = f_x(x_0,y_0,z_0)\Delta x + f_y(x_0,y_0,z_0)\Delta y + f_z(x_0,y_0,z_0)\Delta z + o(\rho)\]
%
where \(\Delta x = x - x_0\), \(\Delta y = y - y_0\), \(\Delta z = z - z_0\), and \(\rho = \sqrt{(\Delta x)^2 + (\Delta y)^2 + (\Delta z)^2}\).
\end{mydef}

\begin{mydef}
\normalfont Let \(f\) be a function of the variables \(x_1, x_2, \hdots, x_n\), i.e. 
%
\[f = f(x_1, x_2, \hdots, x_n)\]
%
where each \(x_j\) is a function of (some of) the variables \(t_1, t_2, \hdots, t_m\), i.e. 
%
\[x_j = x_j(t_1, t_2, \hdots, t_m), j = 1, 2, \hdots, n\]
%
If \(f\) and \(x_j\) are sufficiently smooth, then
%
\[\frac{\partial f}{\partial t_i} = \frac{\partial f}{\partial x_1}\frac{\partial x_1}{\partial t_i} + \frac{\partial f}{\partial x_2}\frac{\partial x_2}{\partial t_i} + \hdots + \frac{\partial f}{\partial x_n}\frac{\partial x_n}{\partial t_i}, i = 1,2,\hdots, m\]
%
\end{mydef}

\pagenumbering{arabic}


























